线性规划是数学规划的一个重要分支,它在理论和算法上都比较成熟,在实践上有着广泛的应用,不仅许多实际课题属于线性规划问题,而且运筹学其他分支中的一些问题也可以转化为线性规划来计算,因此线性规划在最优化学科中占有重要地位。
\section{标准形式及图解法}
一般线性规划问题总可以写成下列标准形式:
\begin{equation}
	\begin{aligned}
		min &\quad cx \\
		s.t. &\quad Ax = b,\\
		&\quad  x \geq 0,
	\end{aligned}
\end{equation}
其中$A$是$m \times n$矩阵,$c$是$n$维行向量,$b$是$m$维列向量。
\subsection*{基本性质}
\begin{enumerate}
	\item 可行域。线性规划的可行域是凸集。
	\item 最优极点。设线性规划的可行域非空,则有下列结论:
	\begin{enumerate}
		\item 线性规划存在有限最优解的条件是所有$cd^{(j)}$为非负数。其中$d^{(j)}$是可行域的极方向。
		\item 若线性规划存在有限最优解,则目标函数的最优值可在某个极点上达到。
	\end{enumerate}
	\item 最优基本可行解。极点是个几何概念,有直观性强的优点,但不便于演算,因此需要研究极点的代数含义。
	\begin{equation}
		x = 
		\begin{bmatrix}
		x_B \\ x_N
		\end{bmatrix}
		=
		\begin{bmatrix}
		B^{-1}b \\ 0
		\end{bmatrix}
	\end{equation}
	称为方程组$Ax = b$的一个\textbf{基本解}。B称为基矩阵。简称为基。$x_B$的各分量称为基变量,基变量的全体$x_{B_1},x_{B_2},\dots ,x_{B_m}$称为一组基。$x_N$的各分量称为非基变量。又若$B^{-1} b \geq 0$,则称$x$为约束条件$Ax = b,x \geq 0$的基本可行解。相应地,称$B$为可行基矩阵,$x_{B_1},x_{B_2},\dots ,x_{B_m}$为一组可行基。若$B^{-1}b > 0$,即基变量的取值均为正数,则称基本可行解是非退化的。如果满足$B^{-1}b \geq 0$且至少有一个分量是零,则称基本可行解是退化的基本可行解。
	
	令$K = \{x|Ax = b,x\geq 0\},A$是$m\times n$矩阵,$A$的秩为$m$,则$K$的极点集与$Ax = b,x \geq 0$的基本可行解集等价。
	
	当线性规划存在最优解时,则一定存在一个基本可行解,它是最优解。这样,线性规划问题的求解,可归结为求最优基本可行解。
	\item 基本可行解的存在问题。如果$Ax = b ,x\geq 0$有可行解,则一定存在基本可行解。其中$A$是$m\times n$矩阵,$A$的秩为$m$。
\end{enumerate}
