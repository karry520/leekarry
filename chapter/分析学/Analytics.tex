分析学中最基本的概念是\textbf{极限}。数学和物理中的许多重要概念可以用极限定义,如速度、功、能量、功率、作用、物体的体积和表面积、曲线的长和曲率、曲面的曲率等。
\begin{center}
\begin{tikzpicture}[node distance=2cm]
	\node (a) {极限};
	\node (b)[below of=a,xshift=-2cm] {微分};
	\node (c)[below of=a,xshift=2cm] {积分};
	\node (e)[below of=b]{微分方程};
	\node (f)[below of=c]{变分法和积分方程};
	\node (d)[below of=e,xshift=2cm] {自然科学};

	\draw [->] (a) -- (b);
	\draw [->] (a) -- (c);
	\draw [->] (b) -- (e);
	\draw [->] (c) -- (f);
	\draw [->] (e) -- (d);
	\draw [->] (f) -- (d);
\end{tikzpicture}
\end{center}

分析的核心是\textbf{微积分},它是由牛顿和莱布尼茨分别独立发现的。然而,只有当分析与其他数学学科,如代数学、数论、几何、随机理论与数值理论相互作用时,它才能发挥其真正的作用。
\chapter{初等分析}
\input{./chapter/分析学/初等分析/实数}
\input{./chapter/分析学/初等分析/复数}
\section{在振荡上的应用}

\section{对等式的运算}

\section{对不等式的运算}

\chapter{序列的极限}
\section{基本思想}

\section{实数的希尔伯特(Hilbert)公理}

\section{实数序列}

\section{序列收敛准则}

\chapter{函数的极限}
\section{一个实变量的函数}

\section{度量空间和点集}

\section{多变量函数}

\chapter{一个实变函数的微分法}
\input{./chapter/分析学/一个实变函数的微分法/导数}
\section{链式法则}

\section{递增函数和递减函数}

\section{反函数}

\section{泰勒定理和函数的局部行为}

\section{复值函数}

\chapter{多元实变函数的导数}
\section{偏导数}

\section{弗雷歇导数}

\section{链式法则}

\section{对微分算子的变换的应用}

\section{对函数相关性的应用}

\section{隐函数定理}

\section{逆映射}

\section{n阶变分与泰勒定理}

\section{在误差估计上的应用}

\section{弗雷歇微分}

\chapter{单实变函数的积分}
\section{基本思想}

\section{积分的存在性}

\section{微积分基本定理}

\input{./chapter/分析学/单实变函数的积分/分部积分法}
\input{./chapter/分析学/单实变函数的积分/代换}
\section{无界区间上的积分}

\section{无界函数的积分}

\section{柯西主值}

\section{对弧长的应用}

\section{物理角度的标准推理}

\chapter{多实变量函数的积分}
\section{基本思想}

\section{积分的存在性}

\section{积分计算}

\section{卡瓦列里原理(累次积分)}

\input{./chapter/分析学/多实变量函数的积分/代换}
\section{微积分基本定理(高斯-斯托克斯定理)}

\section{黎曼曲面测度}

\section{分部积分}

\section{曲线坐标}

\section{应用到质心和惯性中点}

\section{依赖于参数的积分}

\chapter{向量代数}
\section{向量的线性组合}

\section{坐标系}

\section{向量的乘法}

\chapter{向量分析与物理学领域}
\section{速度和加速度}

\section{梯度、散度和旋度}

\section{在形变上的应用}

\section{哈密顿算子的运算}

\section{功、势能和积分曲线}

\section{对力学的守恒律的应用}

\section{流、守恒律与高斯积分定理}

\section{环量、闭积分曲线与斯托克斯积分定理}

\section{根据源与涡确定向量场(向量分析的主要定理)}

\section{对电磁学中麦克斯韦方程的应用}

\section{经典向量分析与嘉当微分学的关系}

\chapter{无穷级数}
\section{收敛准则}

\section{无穷级数的运算}

\section{幂级数}

\section{傅里叶级数}

\section{发散级数求和}

\section{无穷乘积}

\chapter{积分变换}
\section{拉普拉斯变换}

\section{傅里叶变换}

\input{./chapter/分析学/积分变换/Z变换}
\chapter{常微分方程}
\section{引导性的例子}

\section{基本概念}

\section{微分方程的分类}

\section{初等解法}

\input{./chapter/分析学/常微分方程/应用}
\section{线性微分方程组和传播子}

\section{稳定性}

\section{边值问题和格林函数}

\section{一般理论}

\chapter{偏微分方程}
\section{数学物理中的一阶方程}

\section{二阶数学物理方程}

\section{特征的作用}

\section{关于唯一性的一般原理}

\section{一般的存在性结果}

\chapter{复变函数}
\section{基本思想}

\section{复数列}

\input{./chapter/分析学/复变函数/微分}
\input{./chapter/分析学/复变函数/积分}
\section{微分式的语言}

\section{函数的表示}

\section{留数计算与积分计算}

\section{映射度}

\section{在代数基本定理上的应用}

\section{双全纯映射和黎曼映射定理}

\section{共形映射的例子}

\section{对调和函数的应用}

\section{在静电学和静磁学上的应用}

\section{解析自延拓与恒等原理}

\section{在欧拉伽马函数上的应用}

\section{椭圆函数和椭圆积分}

\section{模形式与P函数的反演问题}

\section{椭圆积分}

\section{奇异微分方程}

\section{在高斯超几何微分问题上的应用}

\section{在贝塞尔微分方程上的应用}

\section{多复变函数}

