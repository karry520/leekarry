\section{指数族分布}
指数族分布的成员有许多共同的重要性质,并且以某种程度的一般性下讨论这些性质是很有启发性的。参数为$\eta$的变量$x$的指数族分布定义为具有下面形式的概率分布的集合
\begin{equation}
\label{fa}
	p(x|\eta)=h(x)g(\eta)exp\{\eta^Tu(x)\}
\end{equation}
其中$x$可能是标量或者向量,可能是离散的或者是连续的。这里$\eta$被称为概率分布的自然参数(natural parameters),$u(x)$是$x$的某个函数。函数$g(\eta)$可以被看成系数,它确保了概率分布是归一化的,因此满足
\begin{equation}
	\int p(x|\eta)dx =1
\end{equation}
下面看一些例子
\begin{enumerate}
	\item 伯努利分布
	
	\begin{equation}
		\begin{aligned}
			p(x|\mu)&=Bern(x|\mu)=\mu^x(1-\mu)^{1-x}\\
			&=exp\{x\ln \mu +(1-x)\ln (1-\mu) \}\\
			&=(1-\mu)exp\left\{\ln\left(\frac{\mu}{1-\mu} \right)x \right\}
		\end{aligned}
	\end{equation}
	与公式$\ref{fa}$比较,可以看出
	\begin{equation}
		\eta=\ln \left(\frac{\mu}{1-\mu}\right)
	\end{equation}
	从中可以解出$\eta$,得到$\mu=\sigma(\eta)$,其中
	\begin{equation}
		\sigma(\eta)=\frac{1}{1+exp(-\eta)}
	\end{equation}
	被称为logistic sigmoid函数。因此可以使用公式$\ref{fa}$给出的标准形式把伯努利分布写成下面的形式
	\begin{equation}
		p(x|\mu)=\sigma(-\eta)exp(\eta x)
	\end{equation}
	\item 单一观测$x$的多项式分布
	
	\begin{equation}
	\begin{aligned}
		p(x|\mu)&=\prod_{k=1}^{M}\mu_k^{x_k}=exp\left\{\sum_{k=1}^{M}x_k\ln\mu_k \right\}\\
		\mu(x)&=x\\
		h(x)&=1\\
		g(\eta)&=1
	\end{aligned}
	\end{equation}
	注意参数$\eta_k$不是相互独立的,因为参数$\mu_k$要满足下面的限制
	\begin{equation}
		\sum_{k=1}^{M}\mu_k=1
	\end{equation}
	因此给定任意M-1个参数$\mu_k$,剩下的参数就固定了。使用这个限制,这种表达方式下多项式分布变成了
	\begin{equation}
		\begin{aligned}
			exp&\left\{\sum_{k=1}^{M}x_k\ln\mu_k \right\}\\
			&=exp\left\{\sum_{k=1}^{M-1}x_k\ln \mu_k +\left(1-\sum_{k=1}^{M-1}x_k \right)\ln \left(1-\sum_{k=1}^{M-1}\mu_k \right)\right\}\\
			&=exp\left\{\sum_{k=1}^{M-1}x_k\ln \left(\frac{\mu_k}{1-\sum_{j=1}^{M-1}\mu_k} \right)+\ln \left(1-\sum_{k=1}^{M-1}\mu_k \right) \right\}
		\end{aligned}
	\end{equation}
	令
	\begin{equation}
		\ln \left(\frac{\mu_k}{1-\sum_{j}\mu_j}\right)=\eta_k
	\end{equation}
	从中我们可以解出$\mu_k$。首先两侧对k求和,然后整理,回带,可得
	\begin{equation}
		\mu_k=\frac{exp(\eta_k)}{1+\sum_{j}exp(\eta_j)}
	\end{equation}
	这被称为softmax函数,或者归一化指数(normalized exponential)。在这个表达方式的形式下,多项式分布的形式为
	\begin{equation}
		p(x|\eta)=\left(1+\sum_{k=1}^{M-1}exp(\mu_k)\right)^{-1}exp(\mu^Tx)
	\end{equation}
	这是指数族分布的标准形式,其中参数向量$\eta=(\eta_1,\dots,\eta_{M-1},0)^T$。在这个指数族分布中
	\begin{flalign}
		\mu(x)&=x\\
		h(x)&=1\\
		g(\eta)&=\left(1+\sum_{k=1}^{M-1}exp(\eta_k)\right)^{-1}
	\end{flalign}
	\item 高斯分布
	\begin{equation}
		\begin{aligned}
		p(x|\mu,\sigma^2)&=\frac{1}{(2\pi\sigma^2)^{\frac{1}{2}}}exp\left\{-\frac{1}{2\sigma^2}(x-\mu)^2 \right\}\\
		&=\frac{1}{(2\pi\sigma^2)^{\frac{1}{2}}}exp\left\{-\frac{1}{2\sigma^2}x^2+\frac{\mu}{\sigma^2}x-\frac{1}{2\sigma^2}\mu^2 \right\}\\
		\end{aligned}
	\end{equation}
	经过简单的推导后,它可以转化为公式$\ref{fa}$给出的标准指数族分布的形式,其中
	\begin{flalign}
		\eta&=\begin{pmatrix}
		\frac{\mu}{\sigma^2}\\\frac{-1}{2\sigma^2}
		\end{pmatrix}\\
		\mu(x)&=\begin{pmatrix}
		x\\x^2
		\end{pmatrix}\\
		h(x)&=(2\pi)^{\frac{1}{2}}\\
		g(\eta)&=(-2\eta_2)^{\frac{1}{2}}exp\left(\frac{\eta_1^2}{4\eta_2^2}\right)
	\end{flalign}
\end{enumerate}
\subsection*{最大似然与充分统计量}
用最大似然法估计公式$\ref{fa}$给出的一般形式的指数族分布的参数向量$\mu$的问题。
\begin{equation}
	\int h(x)g(\eta)exp\{\eta^Tu(x)\}=1
\end{equation}
对上式的两侧关于$\mu$取梯度,我们有
\begin{equation}
	\begin{aligned}
	\triangledown g(\eta)&\underbrace{\int h(x)exp\{\eta^Tu(x) \}dx}_{1/g(\eta)}+g(\eta)\int h(x)exp\{\eta^Tu(x) \}u(x)dx=0
	\\
	&\Rightarrow -\frac{1}{g(\eta)}\triangledown g(\eta)=\int \underbrace{g(\eta) h(x)exp\{\eta^Tu(x) \}}_{p(x|\eta)}u(x)dx=\mathbb{E}[\mu(x)]\\
	&\Rightarrow -\triangledown \ln g(\eta)=\mathbb{E}[u(x)]
	\end{aligned}
\end{equation}
同理,\textbf{$u(x)$的协方差,可以根据$g(\eta)$的二阶导数表达,对于高阶矩的情形也类似}。因此,如果我们能够对一个来自指数族分布的概率分布进行归一化,那么我们总能够通过简单的求微分的方式找到它的矩。现在考虑一组独立同分布的数据$X=\{x_1,\dots,x_N \}$。对于这个数据集,似然函数为
\begin{equation}
	p(X|\eta)=\left(\prod_{n=1}^{N}h(x_n) \right)g(\eta )^Nexp\left\{\eta^T\sum_{n=1}^{N}u(x_n) \right\}
\end{equation}
令$\ln p(X|\eta)$关于$\eta$的导数等于零,我们可以得到最大似然估计$\mu_{ML}$满足的条件
\begin{equation}
	-\triangledown \ln g(\eta)=\frac{1}{N}\sum_{n=1}^{N}u(x_n)
\end{equation}
原则上可以通过这个方程来得到$\mu_{ML}$。我们看到最大似然估计的解只通过$\sum_{n}u(x_n)$对数据产生依赖,因此这个量被称为分布的充分统计量(sufficient statistic)。我们不需要存储整个数据集本身,只需要存储充分统计量的值即可。
\subsection*{共轭先验}
对于指数族分布的任何成员,都存在一个共轭先验,可以写成下面的形式
\begin{equation}
	p(\eta|\varkappa ,v)=f(\varkappa,v)g(\eta)^vexp\{v\eta^T\varkappa \}
\end{equation}
\subsection*{无信息先验}
在许多情形下,我们可能对分布应该具有的形式几乎完全不知道。这时,我们可以寻找一种形式的先验分布,被称为无信息先验(noninformative prior)。这种先验分布的目的是尽量对后验分布产生尽可能小的影响。有时被称为“让数据自己说话”。
\begin{enumerate}
	\item 位置参数 
	\item 缩放参数 
\end{enumerate}