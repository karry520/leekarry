目前为止,关注点是无监督学习,包括诸如概率密度估计和数据聚类等话题。我们现在开始讨论有监督学习,首先讨论的是回归问题。回归问题的目标是在给定$D$维输入(input)变量$\boldsymbol{x}$的情况下,预测一个或者多个连续目标(target)变量$t$的值。线性回归模型有着可调节的参数,具有线性函数的性质,将会成为本章的关注点。线性回归模型的最简单的形式也是输入变量的线性函数。但是,通过将一组输入变量的非线性函数进行线性组合,我们也可以获得一类更加有用的函数,被称为基函数(basis function)。这样的模型是参数的线性模型,这使得其具有一些简单的分析性质,同时关于输入变量是非线性的。

最简单的方法是,直接建立一个适当的函数$y(\boldsymbol{x})$,对于新的输入$\boldsymbol{x}$,这个函数能够直接给出对应的$t$预测。更一般地,从一个概率的观点来看,我们的目标是对预测分布$p(t|\boldsymbol{x})$建模,因为经表达了对于每个$\boldsymbol{x}$的值,我们对于$t$的值的不确定性。从这个条件概率分布中,对于任意的$\boldsymbol{x}$的新值,我们可以对$t$进行预测,这种方法等同于最小化一个恰当的损失函数的期望值。对于实值变量来说,损失函数的一个通常的选择是平方误差损失,这种情况下最优解由$t$的条件期望给出。