\section{固定基函数的局限性}
线性模型有一些重要的局限性,这使得我们要转而关注更加复杂的模型,例如支持向量机和神经网络。困难的产生主要是因为我们假设了基函数在观测到任何数据之前就被固定下来,而这正是维度灾难问题的一个表现形式。结果,基函数的数量随着输入空间的维度D迅速增长,通常是指数方式的增长。

幸运的是,真实数据集有两个性质,可以帮助我们缓解这个问题。
\begin{enumerate}
	\item 数据向量$\{x_n\}$通常位于一个非线性流形内部。由于输入变量之间的相关性,这个流形本身的维度小于输入空间的维度。如果我们使用局部基函数,那么我们可以让基函数只分布在输入空间中包含数据的区域。这种方法被用在径向基函数网络中,也被用在支持向量机和相关向量机当中。神经网络模型使用可调节的基函数,这些基函数有着sigmoid非线性的性质。神经网络可以通过调节参数,使得在输入空间的区域中基函数会按照数据流形发生变化。
	\item 目标变量可能只依赖于数据流形中的少量可能的方向。利用这个性质,神经网络可以通过选择输入空间中基函数产生响应的方向。
\end{enumerate}