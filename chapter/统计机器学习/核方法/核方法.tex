\section{核方法}
有这样一类模式识别的技术:训练数据点或者它的一个子集在预测阶段仍然保留并且被使用。许多线性参数模型可以被转化为一个等价的“对偶表示”。对偶表示中,预测的基础也是在训练数据点处计算的核函数(kernel function)的线性组合。对于基于固定非线性特征空间(feature space)映射$\phi(\boldsymbol{x})$的模型来说,核函数由下面的关系给出。
\begin{equation}
\label{kernel}
	k(\boldsymbol{x},\boldsymbol{x}^{'})=\phi(\boldsymbol{x})^T\phi(\boldsymbol{x}^{'})
\end{equation}
核的概念由Aizenman引入模型识别领域。那篇文章介绍了势函数的方法。之所以被称为势函数,是因为它类似于静电学中的概念。虽然被忽视了很多年,但是Boser在边缘分类器的问题中把它重新引入到了机器学习领域。那篇文章提出了支持向量机的方法。从那里起,这个话题在理论上和实用上都吸引了大家的兴趣。一个最重要的发展是把核方法进行了扩展,使其能处理符号化的物体,从而极大地扩展了这种方法能处理的问题的范围。