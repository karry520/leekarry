\section{对偶问题}
注意到式$\ref{eq:3}$本身是一个凸二次规划问题,能直接用现成的优化计算包求解,但我们可以有更高效的办法。由于这个问题的特殊结构,还可以通过拉格朗日对偶性变换到对偶变量的优化问题,即通过求解与原问题等价的对偶问题得到原始问题的最优解,这就是线性可分条件下支持向量机的对偶算法,这样做的优点在于:
\begin{enumerate}
	\item 对偶问题往往更容易求解;
	\item 可以自然的引入核函数,进而推广到非线性分类问题。	
\end{enumerate}
该问题的拉格朗日函数可写为
\begin{equation}
	L(w,b,a) = \frac{1}{2}\lVert w \rVert^2 - \sum_{i=1}^na_i (y_i(w^T x_i + b) - 1)
\end{equation}
然后令
\begin{equation}
	\theta (\boldsymbol{w}) = \mathop{max} \limits_{a_i \geq 0}\ L(\boldsymbol{w},b,a)
\end{equation}
具体写出来,目标函数变成了
\begin{equation}
	\mathop{min}\limits_{\boldsymbol{w},b}\ \theta(\boldsymbol{w}) = \mathop{min}\limits_{\boldsymbol{w},b}\mathop{ max}\limits_{a_i \geq 0}\ L(\boldsymbol{w},b,a) = p^*
\end{equation}
这里用$p^*$表示这个问题的最优值,且和最初的问题是等价的。如果直接求解,那么一上来便得面对$w$和$b$两个参数,而$a_i$以是不等式约束,这个求解过程不好做。考虑对偶问题
\begin{equation}
		\mathop{min}\limits_{\boldsymbol{w},b}\ \theta(\boldsymbol{w}) = \mathop{ max}\limits_{a_i \geq 0}\ \mathop{min}\limits_{\boldsymbol{w},b}\ L(\boldsymbol{w},b,a) = d^*
\end{equation}
原始问题通过满足KKT条件,已经转化成了对偶问题。而求解这个对偶问题,分为3个步骤
\begin{enumerate}
	\item \textbf{让$L(\boldsymbol{w},b,a)$关于$\boldsymbol{w}$和$b$最小化}
	
	首先固定$a$,要让$L$关于$w$和$b$最小化,分别对$w$和$b$求偏导,令其等于0;
	\begin{equation}
		\begin{aligned}
		\frac{\partial L}{\partial \boldsymbol{w}} &= \lVert \boldsymbol{w}\rVert - \sum_{i = 1}^n a_iy_ix_i \boldsymbol{w}^T = 0 \Rightarrow  \boldsymbol{w} = \sum_{i = 1}^n a_iy_ix_i  \\
		\frac{\partial L}{\partial b} &= \sum_{i = 1}^n a_iy_i = 0\Rightarrow  \sum_{i = 1}^n a_iy_i  = 0
		\end{aligned}
	\end{equation}
	将以上结果代入之前的L,得到
	\begin{equation}
		\begin{aligned}
		L(\boldsymbol{w},b,a) &= \frac{1}{2}\sum_{i,j=1}^na_ia_jy_iy_jx_i^Tx_j - \sum_{i,j=1}^na_ia_jy_iy_jx_i^Tx_j - b\sum_{i = 1}^n a_iy_i + \sum_{i = 1}^na_i \\
		&= \sum _{i = 1}^n a_i - \frac{1}{2}\sum_{i,j=1}^na_ia_jy_iy_jx_i^Tx_j
		\end{aligned}
	\end{equation}
	\item \textbf{求对$a$的极大}
	
	求对$a$的极大,即是关于对偶问题的最优化问题。经过上一个步骤的求解,得到的拉格朗日函数式子已经没有了变量$w$和$b$,只有$a$。从上面的式子得到
	\begin{equation}
		\begin{aligned}
		\mathop{max}\limits_a \quad &\sum _{i = 1}^n a_i - \frac{1}{2}\sum_{i,j=1}^na_ia_jy_iy_jx_i^Tx_j \\
		s.t. \quad &a_i \geq 0,i = 1,\dots, n \\
		&\sum_{i = 1}^n a_iy_i  = 0
		\end{aligned}
	\end{equation}
	这样,求出了$a_i$,从而根据
	\begin{equation}
		\begin{aligned}
		\boldsymbol{w}^* &= \sum_{i = 1}^n a_iy_ix_i \\
		b^* &= -\frac{\mathop{max}\ {\boldsymbol{w}^*}^T x_i + \mathop{min}\ {\boldsymbol{w}^*}^T x_i}{2} 
		\end{aligned}
	\end{equation}
	即可求出$w,b$,最终得出分离超平面和分类决策函数。
	\item \textbf{利用SMO算法求解对偶问题中的拉格朗日乘子}
	
	在求得$L(w,b,a)$关于$w$和$b$最小化和对$a$的极大之后,最后一步便是利用SMO算法求解对偶问题中的拉格朗日乘子	
\end{enumerate}
