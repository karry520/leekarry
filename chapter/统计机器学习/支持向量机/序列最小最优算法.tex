\section{序列最小最优算法}
接上一小节,讨论支持向量机学习的实现问题。讲述其中的序列最小最优化(sequential minimal optimization,SMO)算法。

SMO算法是一种启发式算法,其基本思路是:\textbf{如果所有变量的解都满足此最优化问题的KKT条件,那么这个最优化问题的解就得到了}。因为KKT条件是该最优化问题的充分必要条件。否则,选择两个变量,固定其他变量,针对这两个变量构建一个二次规划问题。这个二次规划问题关于这两个变量的解应该更接近原始二次规划问题的解,因为这会使得原始二次规划问题的目标函数值变得更小。重要的是,这时子问题可以通过解析方法求解,这样就可以大大提高整个算法的计算计算速度。子问题有两个变量,一个是违反KKT条件最严重的那一个,另一个由约束条件自动确定。如此,SMO算法将原问题不断分解为子问题并对子问题求解,进而达到求解原问题的目的。

SMO算法要解如下凸二次规划的对偶问题:
\begin{flalign}
	\mathop{min}\limits_a \quad &\frac{1}{2}\sum_{i=1}^N\sum_{j=1}^{N}a_ia_jy_iy_jK(x_i, x_j) - \sum _{i = 1}^N a_i\label{eq:最优化问题}\\
	s.t. \quad &C\geq a_i \geq 0,\quad i = 1,\dots, n \\
	&\sum_{i = 1}^N a_iy_i  = 0\label{eq:等式约束}
\end{flalign}
注意,子问题的两个变量中只有一个是自由变量。假设$a_1,a_2$为两个变量,$a_3,a_4,\dots,a_N$固定,那么由等式约束$\ref{eq:等式约束}$可知
\begin{equation}
	a_1 = - y_1 \sum_{i=2}^{N}a_iy_i
\end{equation}
如果$a_2$确定,那么$a_1$也随之确定。所以子问题中同时更新两个变量。

整个SMO算法包括两个部分:\textbf{求解两个变量二次规划的解析方法和选择变量的启发式方法}。

于是SMO的最优化问题$\ref{eq:最优化问题}$的子问题可以写成
\begin{flalign}
	\mathop{min}\limits_{a_1,a_2}\ W(a_1,a_2)&=\frac{1}{2}K_{11}a_1^2+\frac{1}{2}K_{22}a_2^2+y_1y_2K_{12}a_1a_2 \nonumber\\
	&-(a_1+a_2)+y_1a_1\sum_{i=3}^{N}y_ia_iK_{i1}+y_2a_2\sum_{i=3}^{N}y_ia_iK_{i2}\label{拆解后的公式}\\
	s.t\ &a_1y_1+a_2y_2=-\sum_{i=3}^{N}y_ia_i=\varsigma\\
	&0\leq a_i \leq C,\ i=1,2
\end{flalign}
其中,$K_{ij}=K(x_i,x_j),i,j=1,2,\dots,N,\varsigma$是常数。

为了求解两个变量的二次规划问题$\ref{拆解后的公式}$,首先分析约束条件,然后在此约束条件下求极小。由于只有两个变量$(a_1,a_2)$,约束可以用二维空间中的图形表示,如图所示

\vspace{4pt}
\begin{center}
\begin{tikzpicture}
	\draw (0,0) rectangle (4,4);
	\node (a) at (2,-0.5) {$a_2=0$};
	\node at (2,4.5) {$a_2=C$};
	\node at (-0.5,2) {$a_1=0$};
	\node at (4.5,2) {$a_1=C$};
	\node [below of=a]{$y_1\ne y_2\Rightarrow a_1-a_2=k$};
	\draw[domain=1:4] plot (\x,{\x -1});
	\draw[domain=0:3] plot (\x,{\x +1});
	
	\draw[domain=6:9] plot (\x,{-\x +9});
	\draw[domain=7:10] plot (\x,{-\x +11});  
	\draw (6,0) rectangle (10,4);
	\node (b) at (8,-0.5) {};
	\node [below of=b]{$y_1 = y_2\Rightarrow a_1+a_2=k$};
\end{tikzpicture}
\end{center}
不等式约束使得$(a_1,a_2)$在盒子$[0,C]\times[0,C]$内,等式约束使$(a_1,a_2)$在平行盒子的对角线的直线上。因此要求的是目标函数在一条平行于对角线线的线段上的最优值。这使得两个变量的最优化问题成为实质上的单变量的最优化问题,不妨考虑为变量$a_2$的最优化问题。

假设问题$\ref{拆解后的公式}$的初始可行解为$a_1^{old},a_2^{old}$,最优解为$a_1^{new},a_2^{new}$,并且假设在沿着约束方向未经剪辑时$a_2$的最优解为$a_2^{new,unc}$。

引进记号
\begin{flalign}
	g(x)&=\sum_{i=1}^{N}a_iy_iK(x_i,x)+b\\
	v_i&=\sum_{j=3}^{N}y_ia_iK(x_i,x_j)=g(x_i)-\sum_{j=1}^{2}a_jy_jK(x_i,x_j)-b,\ i=1,2
\end{flalign}
目标函数可写成
\begin{equation}
\label{目标函数}
	W(a_1,a_2)=\frac{1}{2}K_{11}a_1^2+\frac{1}{2}K_{22}a_2^2+y_1y_2K_{12}a_1a_2 -(a_1+a_2)+y_1a_1v_1+y_2a_2v_2
\end{equation}
令
\begin{equation}
	E_i=g(x_i)-y_i=\left(\sum_{j=1}^{N}a_jy_jK(x_j,x_i)+b\right)-y_i,\quad i=1,2
\end{equation}
当$i=1,2$时,$E_i$为函数$g(x)$对输入$x_i$的预测值与真实输出$y_i$之差。
由$a_1y_1=\varsigma -a_2y_2$及$y_i^2=1$,可将$a_1$表示为
\begin{equation}
	a_1=(\varsigma-y_2a_2)y_1
\end{equation}
代入式$\ref{目标函数}$,得到只是$a_2$的函数的目标函数,对$a_2$求导数
\begin{equation}
	\frac{\partial W}{\partial a_2}=K_{11}a_2+K_{22}a_2-2K_{12}a_2-K_{11}\varsigma y_2 + K_{12}\varsigma y_2+y_1y_2-1-v_1v_2+y_2v_2
\end{equation}
令其为0,得到 
\begin{equation}
\begin{split}
	(K_{11}+K_{22}-2K_{12})a_2=&y_2(y_2-y_1+\varsigma K_{11}-\varsigma K_{12}+v_1-v2)\\
	=&y2\left[ y_2-y_1+\varsigma K_{11}-\varsigma K_{12} +\left( g(x_1) - \sum_{j=1}^{2}y_ja_jK_{1j}-b \right)\right. \\
	&\left. \left( g(x_2) - \sum_{j=1}^{2}y_ja_jK_{2j}-b \right) \right]
\end{split}
\end{equation}
将$\varsigma=a_1^{old}y_1+a_2^{old}y_2$代入,得到
\begin{equation}
	\begin{split}
	(K_{11}+K_{22}-2K_{12})a_2^{new,unc}&=y_2((K_{11}+K_{22}-2K_{12})a_2^{old}y_2+y_2-y_1+g(x_1)-g(x_2))\\
	&=(K_{11}+K_{22}-2K_{12})a_2^{old}+y_2(E_1-E_2)
	\end{split}
\end{equation}
将$\eta = K_{11}+K_{22}-2K_{12}$代入,于是得到
\begin{equation}
\label{依赖}
	a_2^{new,unc}=a_2^{old}+\frac{y_2(E_1-E_2)}{\eta}
\end{equation}
要使其满足不等式的约束必须将其限制在区间$[L,H]$内,从而得到$a_2^{new}$的表达式
\begin{equation}
	a_2^{new}=
	\begin{cases}
	H,\qquad a_2^{new,unc}>H\\
	a_2^{new,unc},\qquad L\leq a_2^{new,unc}\leq H\\
	L,\qquad a_2^{new,unc}<L
	\end{cases}
\end{equation}
如果$y_1\ne y_2$
\begin{equation*}
	L=\mathop{max}(0,a_2^{old}-a_1^{old}),\quad H=\mathop{min}(C,C+a_2^{old}-a_1^{old})
\end{equation*}
如果$y_1 = y_2$
\begin{equation*}
L=\mathop{max}(0,a_2^{old}+a_1^{old}-C),\quad H=\mathop{min}(C,a_2^{old}+a_1^{old})
\end{equation*}
\textbf{变量的选择方法}

SMO算法在每个子问题中选择两个变量优化,其中至少一个变量是违反KKT条件的。
\begin{enumerate}
	\item 第1个变量的选择
	
	SMO称选择第1个变量的过程为外层循环。外层循环在训练样本中选取违反KKT条件最严重的样本点,并将其对应的变量作为第1个变量。具体地,检验训练样本点$(x_i,y_i)$是否满足KKT条件,即
	\begin{flalign}
		a_i=0&\Leftrightarrow y_ig(x_i)\geq 1\\
		C>a_i>0&\Leftrightarrow y_ig(x_i)= 1\\
		a_i=C&\Leftrightarrow y_ig(x_i)\leq 1
	\end{flalign}
	\item 第2个变量的选择
	
	SMO称选择第2个变量的过程为内层循环,假设在外层循环中已经找到第1个变量$a_1$,现在要在内层循环中找到第2个变量$a_2$。第2个变量选择的标准是希望能使$a_2$有足够大的变化。由式$\ref{依赖}$知,$a_2^{new}$是依赖于$|E_1-E_2|$的,加了加快计算速度,一种简单的做法是选择$a_2$,使其对应的$|E_1-E_2|$最大。
	\item 计算阈值$b$和差值$E_i$
\end{enumerate}
