\section{基本概念}

\subsection*{统计学习的特点}
\textbf{统计学习(statistical learning)}是关于计算机基于数据构建概率统计模型并运用模型对数据进行预测与分析的一门学科。
\begin{itemize}
	\item 统计学习以计算机及网络为平台;
	\item 统计学习以数据为研究对象;
	\item 统计学习的目的是对数据进行预测与分析;
	\item 统计学习以方法为中心;
	\item 统计学习是概率论、统计学、信息论、计算理论、最优化理论及计算机科学等多个领域的交叉学科 。
\end{itemize}

\subsection*{统计学习的对象}
统计学习的对象是数据(data)。它从数据出发,提取数据的特征,抽象出数据的模型,发现数据中的知识,又回到对数据的分析与预测中去。\textbf{统计学习关于数据的基本假设是同类数据具有一定的统计规律性,这是统计学习的前提}。
\subsection*{统计学习的目的}
统计学习用于对数据进行预测与分析,特别是对未知新数据进行预测与分析。对数据的预测与分析是通过构建概率统计模型实现的。统计学习总的目标就是考虑学习什么样的模型和如何学习模型,以使模型能对数据进行准确的预测与分析,同时也要考虑尽可能地提高学习效率。
\subsection*{统计学习的方法}
统计学习的方法是基于数据构建统计模型从而对数据进行预测与分析。
\begin{itemize}
	\item 监督学习(supervised learning)
	\item 非监督学习(unsupervised learning)
	\item 半监督学习(semi-supervised learning)
	\item 强化学习(reinforcement learning)
\end{itemize}
\textbf{实现统计学习方法的步骤如下:}
\begin{enumerate}[(1)]
	\item 得到一个有限的训练数据集合
	\item 确定包含所有可能的模型的假设空间,即学习模型的集合
	\item 确定模型选择的准则,即学习的策略
	\item 实现求解最优化模型的算法,即学习的算法 
	\item 通过学习方法选择最优模型
	\item 利用学习的最优模型对新数据进行预测或分析
\end{enumerate}

\subsection*{统计学习的研究}
统计学习方法(statistical learning method),旨在开发新的学习方法。

统计学习理论(statistical learning theory),旨在探求统计学习方法的有效性与效率。

统计学习应用(application of statistical learning),旨在将统计学习方法应用到实际问题中去,解决实际问题。
\subsection*{统计学习的重要性}
统计学习是处理海量数据的有效方法;
统计学习是计算机智能化的有效手段;
统计学习是计算机科学习发展的一个重要组成部分。




