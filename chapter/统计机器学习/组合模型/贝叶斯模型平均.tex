\section{贝叶斯模型平均}
将模型组合方法与贝叶斯模型平均区分开是很重要的,这两种方法经常被弄混淆。为了理解二者的差异,考虑使用高斯混合模型进行概率密度估计的例子,其中若干的高斯分量以概率的方式进行组合。
\begin{equation}
	p(\boldsymbol{x})=\sum_{k=1}^{K}\pi_k\mathcal{N}(\boldsymbol{x}|\boldsymbol{\mu}_k,\Sigma_k)
\end{equation}
这是模型组合的一个例子。

现在假设我们有若干个不同的模型,索引为$h=1,\dots,H$,先验概率分布为$p(h)$。例如一个模型可能是高斯混合模型,另一个模型可能是西分布的混合。数据集上的边缘概率分布为
\begin{equation}
	p(\boldsymbol{X})=\sum_{h=1}^{H}p(\boldsymbol{X}|h)p(h)
\end{equation}
这是贝叶斯模型平均的一个例子。这个在h上的求和式的意义是,只有一个模型用于生成整个数据集,h上的概率分布仅仅反映了我们对于究竟是哪个模型用于生成数据的不确定性。随着数据规模的增加,这个不确定性会减小,后验概率分布$p(h|\boldsymbol{X})$会逐渐集中于模型中的某一个。

这就强调了贝叶斯模型平均和模型组合的一个关键不同,因为在贝叶斯模型平均中,整个数据集由单一的模型生成。相反,当我们组合多个模型时,我们看到数据集中的不同的数据点可以由潜在变量$\boldsymbol{z}$的不同的值生成,即由不同的分量生成。