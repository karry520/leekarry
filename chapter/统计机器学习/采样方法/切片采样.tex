\section{切片采样}
我们已经看到,Metropolis算法的一个困难之处是它对于步长的敏感性。如果步长过小,那么由于随机游走行为,算法会很慢。而如果步长过大,那么由于较高的拒绝率,算法会很低效。切片采样(slice sampling)方法提供了一个可以自动调节步长来匹配分布特征的方法。与之前一样,它需要我们能够计算未归一化的概率分布$\tilde{p}(z)$。

首先考虑一元变量的情形。切片采样涉及到使用额外的变量$u$对$z$进行增广,然后从联合的$(z,u)$空间中采样。目标是从下面的概率分布
\begin{equation}
	\hat{p}(z,u)=\begin{cases}
		\frac{1}{Z_p}\qquad \text{如果}0\leqslant u \leqslant \tilde{p}(z)\\
		0\qquad \text{其他情况}
	\end{cases}
\end{equation}
中均匀地进行采样,其中$Z_p=\int \tilde{p}(z)dz$。$z$上的边缘概率分布为
\begin{equation}
	\int \hat{p}(z,u)du=\int _0^{\tilde{p}(z)}\frac{1}{Z_p}du=p(z)
\end{equation}
因此,我们可能通过从$\hat{p}(z,u)$中采样,然后忽略$u$值的方式得到$p(z)$的样本。通过交替地对$z$和$u$进行采样即可完成这一点。给定$z$的值,我们可以计算$\tilde{p}(z)$的值,然后在$0\leqslant u \leqslant \tilde{p}(z)$上均匀地对$u$进行采样,这很容易。然后,我们固定$u$,在由$\{z:\tilde{p}(z)>u \}$定义的分布的“切片”上,对$z$进行均匀地采样。

在实际应用中,直接从穿过概率分布的切片中采样很困难,因此我们定义了一个采样方法,它保持$\hat{p}(z,u)$下的均匀分布具有不变性,这可以通过确保萍踪细节平衡的套件来实现。假设$z$的当前值记作$z^{(\tau)}$,并且我们已经得到了一个对应的样本$u$。$z$的下一个值可以通过考察包含$z^{(\tau)}$的区域$z_{\mathrm{min}}\leqslant z\leqslant z_{\mathrm{max}}$来获得。根据概率分布的特征长度标度来对步长进行调节就发生在这里。我们希望区域包含尽可能多的切片,从而使得$z$空间中能进行较大的移动,同时希望切片外的区域尽可能小,因为切片外的区域会使得采样变得低效。因为切片外的区域会使得采样变得低效。

一种选择区域的方法是,从一个包含$z^{(\tau)}$的具有某个宽度$w$的区域开始,然后测试每个端点,看它们是否位于切片内部。如果有端点没在切片内部,那么区域在增加$w$值的方向上进行扩展,直到端点位于区域外。然后,$z^{'}$的一个样本被从这个区域中均匀抽取。如果它位于切片内,那么它就构成了$z^{(\tau+1)}$。如果它位于切片外,那么区域收缩,使得$z^{'}$组成一个端点,并且区域仍然包含$z^{(\tau)}$。然后,另一个样本点从这个缩小的区域中均匀抽取,以此类推,直到找到位于切片内部的一个$z$值。

切片采样可以应用于多元分布中,方法是按照吉布斯采样的方式重复地对每个变量进行采样。这要求对于每个元素$z_i$,我们能够计算一个正比于$p(z_i|z_{\backslash i})$的函数。