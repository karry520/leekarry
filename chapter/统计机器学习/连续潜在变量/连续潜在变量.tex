我们讨论了具有离散潜在变量的概率模型,例如高斯混合模型。我们现在研究某些潜在变量或者全部潜在变量为连续变量的模型。研究这种模型的一个重要的动机是许多数据集具有下面的性质:数据点几乎全部位于比原始数据空间的维度低得多的流形中。

在实际应用中,数据点不会被精确限制在一个光滑的低维流形中,我们可以将数据点关于流形的偏移看做噪声。这就自然地引出了这种模型的生成式观点,其中我们首先根据某种潜在变量的概率分布在流形中选择一个点,然后通过添加噪声的方式生成观测数据点。噪声服从给定潜在变量下的数据变量的某个条件概率分布。

最简单的连续潜在变量模型对潜在变量和观测变量都作出了高斯分布的假设,并且使用了观测变量对于潜在变量状态的线性高斯依赖关系。这就引出了一个著名的技术——主成分分析(PCA)的概率表示形式,也引出了一个相关的模型,被称为因子分析。

本章中,我们首先介绍标准的、非概率的PCA方法,然后我们会说明,当求解线性高斯潜在变量模型的一种特别形式的最大似然解时,PCA如何自然地产生。这种概率形式的表示方法会带来很多好处,例如在参数估计时可以使用EM算法,对混合PCA模型的推广,以及主成分的数量可以从数据中自动确定的贝叶斯公式。最后,我们简短地讨论潜在变量概念的几个推广,使得潜在变量的概念不局限于线性高斯假设。这种推广包括非高斯潜在变量,它引出了独立成分分析(independent conponent analysis)的框架。这种推广还包括潜在变量与观测变量的关系是非线性关系的模型。